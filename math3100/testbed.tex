%This template is a fairly basic example of some things you may need to get started writing in LaTeX. 
%-------------------------------------------------------------------------- 
% Welcome to Overleaf! There are a lot of great tutorials on this site that can help you get started. There are also lots of resources (both on this site and out on the web) that can help you do pretty much anything you're trying to do. In particular, TeXStackExchange is a great resource you can use or the LaTeX Wikibook. Just Google "How do I (blah) in latex?" and an answer is almost certain to come up.
%--------------------------------------------------------------------------
% I recommend that you DO NOT touch any of the code that appears in the preamble (i.e before \begin{document}) unless you really know what you're doing. Removing packages could break things you're trying to do. Adding packages usually isn't a big deal, but some packages don't play well with each other. Keep that in mind.
%--------------------------------------------------------------------------

% Beginning of preamble 
% Specify the type of document you want (usually article) and the font size
\documentclass[12pt]{article}  

% Packages to load which give you useful commands
\usepackage{amssymb, amsmath, amsthm, amsfonts}

% Some other packages that I find useful
\usepackage{euscript}
\usepackage{graphicx}
\usepackage[dvipsnames]{xcolor}

% allow for urls and hyperlinks
\usepackage{url}
\usepackage{hyperref}
\hypersetup{colorlinks=true, urlcolor=RoyalBlue, citecolor=RedViolet}

% Set the margins other page formatting
\usepackage{geometry}
    \geometry{left=1in, right=1in, top=1in, bottom=1in}
\usepackage{enumitem}
\usepackage{setspace} 
\setlength\parindent{0pt}
\linespread{1.2}
%\pagestyle{empty}  %This style will suppress page numbering

% Define a few theorem-type environments
\newtheorem*{theorem}{Theorem}
\newtheorem*{problem}{Problem}
\newtheorem*{conjecture}{Conjecture}

\theoremstyle{definition} 
\newtheorem*{definition}{Definition}
\newtheorem*{answer}{Answer}
\newtheorem*{example}{Example}

% Make the font Sans Serif

\renewcommand{\familydefault}{\sfdefault}

% Some notation shortcut commands

\newcommand{\R}{\mathbb{R}} %the reals
\newcommand{\Q}{\mathbb{Q}} %the rationals
\newcommand{\N}{\mathbb{N}} %the natural numbers
\newcommand{\Z}{\mathbb{Z}} %the integers
\newcommand{\eps}{\varepsilon} %a cooler epsilon

\newcommand{\solution}{\textcolor{PineGreen}{Solution:\newline}}

%%%%% End of preamble %%%%%
%--------------------------------------------------------------------------

\begin{document}

Math 3100 \hfill Zachary A. Hampton

\LaTeX~Lab Assignment 2 \hfill 
\today

\bigskip

Section 2.1 Problems 11 (a,b)

\section*{Advanced Math Examples}

\subsection*{1. Calculus: Taylor Series Expansion}

Consider the function $f(x) = e^x$. The Taylor series expansion of $f(x)$ around $x = 0$ is given by:

\[
f(x) = e^x = \sum_{n=0}^{\infty} \frac{x^n}{n!}
\]

To find the Taylor series expansion of $f(x) = \sin(x)$ around $x = 0$, we use:

\[
\sin(x) = \sum_{n=0}^{\infty} \frac{(-1)^n}{(2n+1)!} x^{2n+1}
\]

Applying this, we can approximate $\sin(x)$ using the first few terms:

\[
\sin(x) \approx x - \frac{x^3}{6} + \frac{x^5}{120}
\]

\subsection*{2. Linear Algebra: Eigenvalues and Eigenvectors}

Given a square matrix $A \in \mathbb{R}^{n \times n}$, an eigenvalue $\lambda$ and an eigenvector $v \neq 0$ satisfy:

\[
A v = \lambda v
\]

To compute the eigenvalues of a matrix $A$, we solve the characteristic equation:

\[
\det(A - \lambda I) = 0
\]

For example, for the matrix:

\[
A = \begin{pmatrix}
2 & 1 \\
1 & 2
\end{pmatrix}
\]

The characteristic equation is:

\[
\det\begin{pmatrix}
2 - \lambda & 1 \\
1 & 2 - \lambda
\end{pmatrix} = (2 - \lambda)^2 - 1 = 0
\]

Solving for $\lambda$, we find the eigenvalues $\lambda_1 = 1$ and $\lambda_2 = 3$.

\subsection*{3. Number Theory: The Euclidean Algorithm}

The Euclidean algorithm is a method for finding the greatest common divisor (GCD) of two integers. For two integers $a$ and $b$, the GCD can be found by repeatedly applying the equation:

\[
\text{gcd}(a, b) = \text{gcd}(b, a \mod b)
\]

For example, to find $\text{gcd}(252, 105)$:

\[
252 \div 105 = 2 \quad \text{remainder } 42 \implies \text{gcd}(252, 105) = \text{gcd}(105, 42)
\]
\[
105 \div 42 = 2 \quad \text{remainder } 21 \implies \text{gcd}(105, 42) = \text{gcd}(42, 21)
\]
\[
42 \div 21 = 2 \quad \text{remainder } 0 \implies \text{gcd}(42, 21) = 21
\]

Therefore, $\text{gcd}(252, 105) = 21$.

\subsection*{4. Probability: Bayes' Theorem}

Bayes' theorem describes the probability of an event, based on prior knowledge of conditions that might be related to the event. It is given by:

\[
P(A|B) = \frac{P(B|A) \cdot P(A)}{P(B)}
\]

where:
- $P(A|B)$ is the probability of event $A$ given event $B$ is true,
- $P(B|A)$ is the probability of event $B$ given event $A$ is true,
- $P(A)$ and $P(B)$ are the probabilities of events $A$ and $B$, respectively.

\end{document}

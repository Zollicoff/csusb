%  DO NOT touch any of the code that appears in the preamble (i.e before \begin{document}). Removing packages could break things you're trying to do. Adding packages usually isn't a big deal, but some packages don't play well with each other. Keep that in mind.
%--------------------------------------------------------------------------

% Beginning of preamble 
% Specify the type of document you want (usually article) and the font size
\documentclass[12pt]{article}  

% Packages to load which give you useful commands
\usepackage{amssymb, amsmath, amsthm, amsfonts}

% Some other packages that I find useful
\usepackage{euscript}
\usepackage{graphicx}
\usepackage[dvipsnames]{xcolor}

% allow for urls and hyperlinks
\usepackage{url}
\usepackage{hyperref}
\hypersetup{colorlinks=true, urlcolor=RoyalBlue, citecolor=RedViolet}

% Set the margins other page formatting
\usepackage{geometry}
\geometry{left=1in, right=1in, top=1in, bottom=1in}
\usepackage{enumitem}
\usepackage{setspace}
\setlength\parindent{0pt}
\linespread{1.2}

% Define a few theorem-type environments
\newtheorem*{theorem}{Theorem}
\newtheorem*{prop}{Proposition}
\newtheorem*{cor}{Corollary}
\newtheorem*{lemma}{Lemma}
\newtheorem*{problem}{Problem}
\theoremstyle{definition}
\newtheorem*{definition}{Definition}
\newtheorem*{example}{Example}

% Make the font Helvetica (Best font ever)
\usepackage{helvet}
\renewcommand{\familydefault}{\sfdefault}

% Some notation shortcut commands
\newcommand{\R}{\mathbb{R}} %the reals
\newcommand{\Q}{\mathbb{Q}} %the rationals
\newcommand{\N}{\mathbb{N}} %the natural numbers
\newcommand{\Z}{\mathbb{Z}} %the integers
\newcommand{\eps}{\varepsilon} %a cooler epsilon

%%%%% Edit the information below to include your name. For future LaTeX Lab Assignments, change the lab number

\title{Math 3100 Lab 5}
\author{Zachary A. Hampton}
\date{Fall 2024}

%%%%% End of preamble %%%%%
%--------------------------------------------------------------------------
\begin{document}

%%%%% Prints title page - do not change %%%%%
\maketitle

%%%%% Prints table of contents - do not change %%%%%
\tableofcontents

%%%%% Begin body - See Lab Directions on Canvas %%%%%
\section{Problem 1}

\subsection{Introductory Material}

%%%%% Define Even %%%%%
\begin{definition}[Even]
An integer \( a \) is an \textbf{even integer} provided that there exists an integer \( n \) such that \( a = 2n \).  \cite[pg 16]{Sund21}
\end{definition}

%%%%% Define Odd %%%%%
\begin{definition}[Odd]
An integer \( a \) is an \textbf{odd integer} provided there exists an integer \( n \) such that \( a = 2n + 1 \). \cite[pg 16]{Sund21}
\end{definition}

\subsection{Final Draft}
%%%% Insert the theorem statement in the environment below %%%%
\begin{theorem}
Let \(x\) and \(y\) be integers. If \(x\) is even and \(y\) is odd, then \(xy\) is even.
\end{theorem}

%%%% Insert the proof in the environment below %%%%
\begin{proof}
First, we assume \( x \) is an even integer and \( y \) is an odd integer, and we will prove that the product \( xy \) is an even integer. By the definition of even, there exists an integer \( k \) such that \( x = 2k \). And, by the definition of odd, there exists an integer \( j \) such that \( y = 2j + 1 \). Then, substituting for \( x \) and \( y \) we get,
\begin{align*}
xy &= (2k)(2j+1) \\
   &= 2k(2j+1) \\
   &= 2(k(2j+1)) \\
   &= 2q.
\end{align*}
Since integers are closed under multiplication and addition, \( k(2j+1) \) is an integer. Since \( xy = 2q \) for some integer \( q \), then \( xy \) is an even integer. Therefore, we have proven that assuming \( x \) is even and \( y \) is odd, then the product \( xy \) is even.
\end{proof}

\vspace{1em}
\begin{thebibliography}{9}
\bibitem{Sund21} Sundstrom, T., \emph{Mathematical Reasoning: Writing and Proof}, Version 3,  Creative Commons, 2020.

\bibitem{Classwork} The work on this problem is the result of a class collaboration in which I was an active participant.

\end{thebibliography}

%%%%% End body - do not edit after this line %%%%%
%%%%% Do not delete \end{document} %%%%%

 \end{document}

% Beginning of preamble 
% Specify the type of document you want (usually article) and the font size
\documentclass[12pt]{article}  

% Packages to load which give you useful commands
\usepackage{amssymb, amsmath, amsthm, amsfonts}

% Some other packages that I find useful
\usepackage{euscript}
\usepackage{graphicx}
\usepackage[dvipsnames]{xcolor}

% allow for urls and hyperlinks
\usepackage{url}
\usepackage{hyperref}
\hypersetup{colorlinks=true, urlcolor=RoyalBlue, citecolor=RedViolet}

% Set the margins other page formatting
\usepackage{geometry}
\geometry{left=1in, right=1in, top=1in, bottom=1in}
\usepackage{enumitem}
\usepackage{setspace} 
\setlength\parindent{0pt}
\linespread{1.2}
%\pagestyle{empty}  %This style will suppress page numbering

% Define a few theorem-type environments
\newtheorem*{theorem}{Theorem}
\newtheorem*{problem}{Problem}
\newtheorem*{conjecture}{Conjecture}

\theoremstyle{definition} 
\newtheorem*{definition}{Definition}
\newtheorem*{answer}{Answer}
\newtheorem*{example}{Example}

% Make the font Helvetica (Best font ever)
\usepackage{helvet}
\renewcommand{\familydefault}{\sfdefault}

% Some notation shortcut commands
\newcommand{\R}{\mathbb{R}} %the reals
\newcommand{\Q}{\mathbb{Q}} %the rationals
\newcommand{\N}{\mathbb{N}} %the natural numbers
\newcommand{\Z}{\mathbb{Z}} %the integers
\newcommand{\eps}{\varepsilon} %a cooler epsilon
\newcommand{\solution}{\textcolor{PineGreen}{Solution:\newline}}

%%%%% End of preamble %%%%%
%--------------------------------------------------------------------------

\begin{document}

Math 3100 \hfill Zachary A. Hampton

\LaTeX~Assignment 3 \hfill 
\text 09-23-2024

\bigskip

\begin{theorem}
    $(P \lor Q) \rightarrow R \equiv (P \rightarrow R) \land (Q \rightarrow R)$
\end{theorem}
    
\solution 
\begin{proof}
    Let $P, Q$, and $R$ be statements. We begin with the left side and work towards the right side:
    \begin{align*}
        (P \lor Q) \rightarrow R &\equiv \neg (P \lor Q) \lor R  && \text{[Conditional as Disjunction]} \\
        &\equiv (\neg P \land \neg Q) \lor R  && \text{[De Morgan's Law]} \\
        &\equiv (\neg P \lor R) \land (\neg Q \lor R)  && \text{[Distribution]} \\
        &\equiv (P \rightarrow R) \land (Q \rightarrow R)  && \text{[Conditional as Disjunction]}            
    \end{align*}
    Therefore, $(P \lor Q) \rightarrow R \equiv (P \rightarrow R) \land (Q \rightarrow R)$.
\end{proof}

\textbf{Reflection:}
    \begin{itemize}
        \item I worked from the left side and found the right.
        \item Remembering the problem we did in-class, the left side jumped out instantly at me.
        \item I did not get stuck, it just takes some time.
        \item I did not use Dr. Johnson's solution or help from any other source other than The Summary of Useful Logical Equivalences and my notes; I did however use ChatGPT to help clean up my Latex code.
    \end{itemize}

\end{document}

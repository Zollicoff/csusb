% Beginning of preamble 
% Specify the type of document you want (usually article) and the font size
\documentclass[12pt]{article}  

% Packages to load which give you useful commands
\usepackage{amssymb, amsmath, amsthm, amsfonts}

% Some other packages that may be useful
\usepackage{euscript}
\usepackage{graphicx}
\usepackage[dvipsnames]{xcolor}

% Allow for URLs and hyperlinks
\usepackage{url}
\usepackage{hyperref}
\hypersetup{colorlinks=true, urlcolor=RoyalBlue, citecolor=RedViolet}

% Set the margins and other page formatting
\usepackage{geometry}
\geometry{left=1in, right=1in, top=1in, bottom=1in}
\usepackage{enumitem}
\usepackage{setspace} 
\setlength\parindent{0pt}
\linespread{1.2}
%\pagestyle{empty}  %This style will suppress page numbering

% Define a few custom environments with numbering
\newtheorem{problem}{Problem} 
\newtheorem*{theorem}{Theorem}
\newtheorem*{conjecture}{Conjecture}
\theoremstyle{definition} 
\newtheorem*{definition}{Definition}
\newtheorem*{answer}{Answer}
\newtheorem*{example}{Example}

% Set the font to Helvetica
\usepackage{helvet}
\renewcommand{\familydefault}{\sfdefault}

% Custom notation shortcut commands (currently unused)
\newcommand{\Z}{\mathbb{Z}}
% \newcommand{\R}{\mathbb{R}}
% \newcommand{\Q}{\mathbb{Q}}
% \newcommand{\N}{\mathbb{N}}
% \newcommand{\eps}{\varepsilon}
% \newcommand{\solution}{\textcolor{PineGreen}{Solution:\newline}}

%%%%% End of preamble %%%%%
%--------------------------------------------------------------------------

\begin{document}

Math 3100 \hfill Zachary A. Hampton

Problem Set 8 \hfill 
\text{Due Date: November 2, 2024}

\bigskip

\section*{Problem 1}

\begin{problem}
Consider the following proposition: If \( n \) is an integer, then \( 3 \mid (n^3 - n) \).

\begin{enumerate}[label=(\alph*)]
    \item Write a paragraph proof for the third case of this proposition without using properties of congruence.
    
    \textbf{Solution:} To prove \( 3 \mid (n^3 - n) \), we examine the form \( n = 3k + 2 \) for some integer \( k \), as we assume two other cases (for \( n = 3k \) and \( n = 3k + 1 \)) were covered previously.
    \[
    n^3 - n = (3k + 2)^3 - (3k + 2).
    \]
    Expanding \( (3k + 2)^3 \): 
    \[
    (3k + 2)^3 = 27k^3 + 54k^2 + 36k + 8,
    \]
    so
    \[
    n^3 - n = 27k^3 + 54k^2 + 36k + 8 - (3k + 2) = 27k^3 + 54k^2 + 33k + 6.
    \]
    Each term in this expression is divisible by 3, so we conclude \( 3 \mid (n^3 - n) \).

    \item Briefly explain why this proposition is equivalent to the statement \( n^3 \equiv n \pmod{3} \).
    
    \textbf{Solution:} The statement \( 3 \mid (n^3 - n) \) implies that \( n^3 - n \) is divisible by 3. This is equivalent to saying \( n^3 \equiv n \pmod{3} \) because both expressions ensure that \( n^3 \) and \( n \) yield the same remainder modulo 3.

    \item Using congruences, what are the three cases to consider for \( n \)?

    \textbf{Solution:} The three cases for \( n \) modulo 3 are:
    \begin{enumerate}
        \item \( n \equiv 0 \pmod{3} \)
        \item \( n \equiv 1 \pmod{3} \)
        \item \( n \equiv 2 \pmod{3} \)
    \end{enumerate}
    
    \item Construct a know-show table for the case where \( n \equiv 2 \pmod{3} \) using congruences.
    
    \begin{center}
        \begin{tabular}{|p{.1\textwidth}|p{.5\textwidth}|p{.3\textwidth}|}
        \hline
        \textbf{Step} & \textbf{Know} & \textbf{Reason} \\
        \hline
        Step 1 & Assume \( n \equiv 2 \pmod{3} \) & Given case \\
        \hline
        Step 2 & \( n^3 \equiv 8 \equiv 2 \pmod{3} \) & Calculation of \( n^3 \) in terms of mod 3 \\
        \hline
        Step 3 & Conclude \( n^3 \equiv n \pmod{3} \) & Matches \( n \equiv 2 \pmod{3} \) \\
        \hline
        \end{tabular}
    \end{center}
\end{enumerate}
\end{problem}

\bigskip

\section*{Problem 2}

\begin{problem}
Consider the proposition: Let \( n \in \mathbb{Z} \). If \( n \not\equiv 0 \pmod{5} \), then \( n^2 \equiv 1 \pmod{5} \) or \( n^2 \equiv 4 \pmod{5} \).

\begin{enumerate}[label=(\alph*)]
    \item Explore the theorem by completing the following chart.
    
    \begin{center}
        \begin{tabular}{|p{.1\textwidth}|p{.3\textwidth}|p{.3\textwidth}|}
        \hline
        \( n \) & \( n^2 \) & Does \( n^2 \equiv 1 \) or \( 4 \pmod{5} \)? \\
        \hline
        \( \pm1 \) & 1 & Yes, \( 1 \equiv 1 \pmod{5} \) \\
        \( \pm2 \) & 4 & Yes, \( 4 \equiv 4 \pmod{5} \) \\
        \( \pm3 \) & 9 & Yes, \( 9 \equiv 4 \pmod{5} \) \\
        \( \pm4 \) & 16 & Yes, \( 16 \equiv 1 \pmod{5} \) \\
        \hline
        \end{tabular}
    \end{center}
    
    \item Patterns observed: 
        \begin{itemize}
            \item \( n^2 \) for values not divisible by 5 are either congruent to 1 or 4 modulo 5.
        \end{itemize}
    
    \item The cases to consider based on the Division Algorithm are:
    \begin{enumerate}
        \item \( n \equiv 1 \pmod{5} \)
        \item \( n \equiv 2 \pmod{5} \)
        \item \( n \equiv 3 \pmod{5} \)
        \item \( n \equiv 4 \pmod{5} \)
    \end{enumerate}

    \item Know-show table for \( n \equiv 3 \pmod{5} \).
    
    \begin{center}
        \begin{tabular}{|p{.1\textwidth}|p{.5\textwidth}|p{.3\textwidth}|}
        \hline
        \textbf{Step} & \textbf{Know} & \textbf{Reason} \\
        \hline
        Step 1 & Assume \( n \equiv 3 \pmod{5} \) & Given \\
        \hline
        Step 2 & \( n^2 \equiv 9 \equiv 4 \pmod{5} \) & Calculation \\
        \hline
        \end{tabular}
    \end{center}
\end{enumerate}
\end{problem}

\bigskip

\section*{Problem 3}

\begin{problem}
Consider the following proposition:

\textbf{Proposition:} If \( x \) and \( y \) are real numbers, then
\[
\max(x, y) = \frac{|x - y| + x + y}{2}.
\]

Note: For \( x, y \in \mathbb{R} \), the function \( \max(x, y) \) is defined to output the larger of the two values if \( x \neq y \), and the value \( x \) if \( x = y \).

\begin{enumerate}[label=(\alph*)]

\item Explore the proposition by filling in the chart below:

\begin{center}
    \begin{tabular}{|c|c|c|}
    \hline
    \( (x, y) \) & \( \max(x, y) \) & \( \frac{|x - y| + x + y}{2} \) \\
    \hline
    \( (-3, 10) \) & 10 & \( \frac{| -3 - 10 | + (-3) + 10}{2} = \frac{13 + 7}{2} = 10 \) \\
    \( (-3, -10) \) & -3 & \( \frac{| -3 + 10 | + (-3) + (-10)}{2} = \frac{7 - 13}{2} = -3 \) \\
    \( (0, 5) \) & 5 & \( \frac{| 0 - 5 | + 0 + 5}{2} = \frac{5 + 5}{2} = 5 \) \\
    \( \left(\frac{2}{3}, \frac{12}{18}\right) \) & \( \frac{2}{3} \) & \( \frac{\left| \frac{2}{3} - \frac{2}{3} \right| + \frac{2}{3} + \frac{2}{3}}{2} = \frac{0 + 2/3}{2} = \frac{2}{3} \) \\
    \( (\sqrt{2}, \sqrt{3}) \) & \( \sqrt{3} \) & \( \frac{| \sqrt{2} - \sqrt{3} | + \sqrt{2} + \sqrt{3}}{2} = \sqrt{3} \) \\
    \hline
    \end{tabular}
\end{center}

\item Using the definition of absolute value, there are two cases to consider when proving this equation algebraically. These cases are:

\begin{itemize}
    \item Case 1: When \( x \geq y \), which implies \( |x - y| = x - y \).
    \item Case 2: When \( x < y \), which implies \( |x - y| = y - x \).
\end{itemize}

\item Show the thinking work for a proof by cases of the proposition above.

\begin{center}
    \begin{tabular}{|c|p{10cm}|}
    \hline
    \textbf{Case} & \textbf{Steps} \\
    \hline
    Case 1 & Assume \( x \geq y \). Then \( |x - y| = x - y \). We have:
    \[
    \frac{|x - y| + x + y}{2} = \frac{(x - y) + x + y}{2} = \frac{2x}{2} = x.
    \]
    Since \( x \geq y \), \( \max(x, y) = x \), and the proposition holds. \\
    \hline
    Case 2 & Assume \( x < y \). Then \( |x - y| = y - x \). We have:
    \[
    \frac{|x - y| + x + y}{2} = \frac{(y - x) + x + y}{2} = \frac{2y}{2} = y.
    \]
    Since \( y > x \), \( \max(x, y) = y \), and the proposition holds. \\
    \hline
    \end{tabular}
\end{center}

\item Write a formal proof of the proposition using a proof by cases.

\textbf{Proof:} Let \( x, y \in \mathbb{R} \). We will prove the proposition by considering two cases based on the relationship between \( x \) and \( y \).


\textbf{Case 1:} Assume \( x \geq y \). Then \( |x - y| = x - y \), so we calculate:
\[
\begin{aligned}
    \frac{|x - y| + x + y}{2} &= \frac{(x - y) + x + y}{2} \\
    &= \frac{2x}{2} \\
    &= x.
\end{aligned}
\]
Since \( x \geq y \), it follows that \( \max(x, y) = x \). Thus, the expression \( \frac{|x - y| + x + y}{2} \) is equal to \( \max(x, y) \) for this case.

\textbf{Case 2:} Assume \( x < y \). Then \( |x - y| = y - x \), so we calculate:
\[
\begin{aligned}
    \frac{|x - y| + x + y}{2} &= \frac{(y - x) + x + y}{2} \\
    &= \frac{2y}{2} \\
    &= y.
\end{aligned}
\]
Since \( y > x \), it follows that \( \max(x, y) = y \). Thus, the expression \( \frac{|x - y| + x + y}{2} \) is equal to \( \max(x, y) \) for this case.

In both cases, we have shown that \( \max(x, y) = \frac{|x - y| + x + y}{2} \). Therefore, the proposition holds for all \( x, y \in \mathbb{R} \).


\end{enumerate}
\end{problem}

\end{document}

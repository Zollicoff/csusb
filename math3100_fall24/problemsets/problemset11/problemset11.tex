% Beginning of preamble 
% Specify the type of document you want (usually article) and the font size
\documentclass[12pt]{article}  

% Packages to load which give you useful commands
\usepackage{amssymb, amsmath, amsthm, amsfonts}

% Some other packages that may be useful
\usepackage{euscript}
\usepackage{graphicx}
\usepackage[dvipsnames]{xcolor}

% Allow for URLs and hyperlinks
\usepackage{url}
\usepackage{hyperref}
\hypersetup{colorlinks=true, urlcolor=RoyalBlue, citecolor=RedViolet}

% Set the margins and other page formatting
\usepackage{geometry}
\geometry{left=1in, right=1in, top=1in, bottom=1in}
\usepackage{enumitem}
\usepackage{setspace} 
\setlength\parindent{0pt}
\linespread{1.2}
%\pagestyle{empty}  %This style will suppress page numbering

% Define a few custom environments with numbering
\newtheorem{problem}{Problem} 
\newtheorem*{theorem}{Theorem}
\newtheorem*{conjecture}{Conjecture}
\theoremstyle{definition} 
\newtheorem*{definition}{Definition}
\newtheorem*{answer}{Answer}
\newtheorem*{example}{Example}

% Set the font to Helvetica
\usepackage{helvet}
\renewcommand{\familydefault}{\sfdefault}

% Custom notation shortcut commands (currently unused)
\newcommand{\Z}{\mathbb{Z}}
% \newcommand{\R}{\mathbb{R}}
% \newcommand{\Q}{\mathbb{Q}}
% \newcommand{\N}{\mathbb{N}}
% \newcommand{\eps}{\varepsilon}
% \newcommand{\solution}{\textcolor{PineGreen}{Solution:\newline}}

%%%%% End of preamble %%%%%
%--------------------------------------------------------------------------

\begin{document}

Math 3100 \hfill Zachary Hampton

Problem Set 11 \hfill Due Date: 11-29-2024

\bigskip

\section*{Problem 1}

\begin{problem}
Let $A$, $B$, $C$, and $D$ be subsets of a universal set $U$.

\begin{enumerate}[label=(\alph*)]
    \item Complete the following sentence (write the full sentence in your work):

    We say that $(A - B) \subseteq (C \cap D)$ provided that for each $x \in U$, if $x \in (A - B)$, then \underline{\hspace{4cm}}.

    \item Write a useful negation of the statement from part (a).

    \item Write the contrapositive of the statement from part (a).
\end{enumerate}
\end{problem}

\textbf{Solution:}

\begin{enumerate}[label=(\alph*)]
    \item We say that $(A - B) \subseteq (C \cap D)$ provided that for each $x \in U$, if $x \in (A - B)$, then \textbf{$x \in C$ and $x \in D$}.

    \item The negation of the statement is:

    There exists an $x \in U$ such that $x \in (A - B)$ and \textbf{($x \notin C$ or $x \notin D$)}.

    \item The contrapositive of the statement is:

    For each $x \in U$, if \textbf{$x \notin C$ or $x \notin D$}, then $x \notin (A - B)$.

    Since $x \notin (A - B)$ means that $x \notin A$ or $x \in B$, we can also express it as:

    For each $x \in U$, if $x \notin C$ or $x \notin D$, then $x \notin A$ or $x \in B$.
\end{enumerate}

\newpage

\section*{Problem 2}

\begin{problem}
Let $S = \{\, x \in \mathbb{R} \mid x^2 < 9 \,\}$ and $T = \{\, x \in \mathbb{R} \mid x < 3 \,\}$.

\begin{enumerate}[label=(\alph*)]
    \item Is $S \subseteq T$? Justify your conclusion with a formal proof or a formal counterexample argument.

    \item Is $T \subseteq S$? Justify your conclusion with a formal proof or a formal counterexample argument.
\end{enumerate}
\end{problem}

\textbf{Solution:}

\begin{enumerate}[label=(\alph*)]
    \item \textbf{Yes, $S \subseteq T$.}

    \textit{Proof:}

    Let $x \in S$. Then by definition, $x^2 < 9$.

    This implies that $-3 < x < 3$.

    Therefore, $x < 3$.

    Hence, $x \in T$.

    Since every element of $S$ is also in $T$, we conclude that $S \subseteq T$.

    \item \textbf{No, $T \nsubseteq S$.}

    \textit{Counterexample:}

    Consider $x = -4$.

    We have $x = -4 < 3$, so $x \in T$.

    However, $x^2 = (-4)^2 = 16 \geq 9$, so $x \notin S$.

    Therefore, $T \nsubseteq S$.
\end{enumerate}

\newpage

\section*{Problem 3}

\begin{problem}
Let $A = \{\, x \in \mathbb{Z} \mid x \equiv 1 \pmod{5} \,\}$ and $B = \{\, y \in \mathbb{Z} \mid y \equiv 7 \pmod{10} \,\}$.

\begin{enumerate}[label=(\alph*)]
    \item List at least five different elements of the set $A$ and at least five different elements of the set $B$. Each list should contain at least one negative integer. Use proper set notation to write your lists.

    \item Write a formal proof to show that $A$ and $B$ are disjoint (i.e., $A \cap B = \emptyset$).
\end{enumerate}
\end{problem}

\textbf{Solution:}

\begin{enumerate}[label=(\alph*)]
    \item Elements of $A$:

    Since $x \equiv 1 \pmod{5}$, $x = 5k + 1$ for some $k \in \mathbb{Z}$.

    Examples:

    \[
    \begin{aligned}
    &k = -3: \quad x = 5(-3) + 1 = -14 \\
    &k = -2: \quad x = 5(-2) + 1 = -9 \\
    &k = -1: \quad x = 5(-1) + 1 = -4 \\
    &k = 0: \quad x = 5(0) + 1 = 1 \\
    &k = 1: \quad x = 5(1) + 1 = 6
    \end{aligned}
    \]

    Therefore, a subset of $A$ is:

    \[
    A = \{ -14,\ -9,\ -4,\ 1,\ 6,\ \ldots \}
    \]

    Elements of $B$:

    Since $y \equiv 7 \pmod{10}$, $y = 10m + 7$ for some $m \in \mathbb{Z}$.

    Examples:

    \[
    \begin{aligned}
    &m = -2: \quad y = 10(-2) + 7 = -13 \\
    &m = -1: \quad y = 10(-1) + 7 = -3 \\
    &m = 0: \quad y = 10(0) + 7 = 7 \\
    &m = 1: \quad y = 10(1) + 7 = 17 \\
    &m = 2: \quad y = 10(2) + 7 = 27
    \end{aligned}
    \]

    Therefore, a subset of $B$ is:

    \[
    B = \{ -13,\ -3,\ 7,\ 17,\ 27,\ \ldots \}
    \]

    \item \textit{Proof that $A \cap B = \emptyset$:}

    Suppose, for contradiction, that there exists an integer $x$ such that $x \in A \cap B$.

    Then:

    \[
    \begin{cases}
    x \equiv 1 \pmod{5} \\
    x \equiv 7 \pmod{10}
    \end{cases}
    \]

    Since $x \equiv 1 \pmod{5}$, it means $x \mod 5 = 1$.

    Since $x \equiv 7 \pmod{10}$, we have $x \mod 10 = 7$.

    Observe that:

    \[
    x \mod 5 = (x \mod 10) \mod 5.
    \]

    Compute $(x \mod 10) \mod 5$:

    \[
    (7 \mod 5) = 2.
    \]

    Therefore:

    \[
    x \mod 5 = 2.
    \]

    But this contradicts the earlier conclusion that $x \mod 5 = 1$.

    Therefore, no such $x$ exists, and $A \cap B = \emptyset$.
\end{enumerate}

\newpage

\section*{Problem 4}

\begin{problem}
Read the proof below. For each algebraic step, provide justification by citing the appropriate parts of Theorem 5.18 and Theorem 5.20.

\textbf{Proposition:} If $A$, $B$, and $C$ are subsets of some universal set $U$, then:

\[
A - (B \cup C) = (A - B) \cap (A - C).
\]

\textbf{Proof:} Let $A$, $B$, and $C$ be subsets of some universal set $U$. Beginning with the left side, we have:

\begin{align*}
A - (B \cup C) &= A \cap (B \cup C)^{c} \quad \\
&= A \cap (B^{c} \cap C^{c}) \quad \\
&= (A \cap B^{c}) \cap C^{c} \quad \\
&= (A \cap B^{c}) \cap (A \cap C^{c}) \quad \\
&= (A - B) \cap (A - C) \quad 
\end{align*}

Therefore, \[A - (B \cup C) = (A - B) \cap (A - C)\] for any sets $A$, $B$, and $C$.
\end{problem}

\textbf{Solution:}
\begin{align*}
\text{Step 1:}\quad A - (B \cup C) &= A \cap (B \cup C)^{c} \quad \text{[Definition of Set Difference]} \\
\text{Step 2:}\quad &= A \cap (B^{c} \cap C^{c}) \quad \text{[De Morgan's Law]} \\
\text{Step 3:}\quad &= (A \cap B^{c}) \cap C^{c} \quad \text{[Associative Law of Intersection]} \\
\text{Step 4:}\quad &= (A \cap B^{c}) \cap (A \cap C^{c}) \quad \text{[Distributive Law]} \\
\text{Step 5:}\quad &= (A - B) \cap (A - C) \quad \text{[Definition of Set Difference]}
\end{align*}

\newpage

\end{document}

% Beginning of preamble 
% Specify the type of document you want (usually article) and the font size
\documentclass[12pt]{article}  

% Packages to load which give you useful commands
\usepackage{amssymb, amsmath, amsthm, amsfonts}

% Some other packages that I find useful
\usepackage{euscript}
\usepackage{graphicx}
\usepackage[dvipsnames]{xcolor}

% allow for urls and hyperlinks
\usepackage{url}
\usepackage{hyperref}
\hypersetup{colorlinks=true, urlcolor=RoyalBlue, citecolor=RedViolet}

% Set the margins other page formatting
\usepackage{geometry}
\geometry{left=1in, right=1in, top=1in, bottom=1in}
\usepackage{enumitem}
\usepackage{setspace} 
\setlength\parindent{0pt}
\linespread{1.2}
%\pagestyle{empty}  %This style will suppress page numbering

% Define a few theorem-type environments with numbering
\newtheorem{problem}{Problem} % Automatically numbers each problem
\newtheorem*{theorem}{Theorem}
\newtheorem*{conjecture}{Conjecture}
\theoremstyle{definition} 
\newtheorem*{definition}{Definition}
\newtheorem*{answer}{Answer}
\newtheorem*{example}{Example}

% Make the font Helvetica (Best font ever)
\usepackage{helvet}
\renewcommand{\familydefault}{\sfdefault}

% Some notation shortcut commands
\newcommand{\R}{\mathbb{R}} %the reals
\newcommand{\Q}{\mathbb{Q}} %the rationals
\newcommand{\N}{\mathbb{N}} %the natural numbers
\newcommand{\Z}{\mathbb{Z}} %the integers
\newcommand{\eps}{\varepsilon} %a cooler epsilon
\newcommand{\solution}{\textcolor{PineGreen}{Solution:\newline}}

%%%%% End of preamble %%%%%
%--------------------------------------------------------------------------

\begin{document}

Math 3100 \hfill Zachary A. Hampton

\LaTeX~for the fun of it! \hfill 
\text{10-25-2024}

\bigskip

\section*{Problem Set 7}

\begin{problem}
Let \( n \in \mathbb{N} \) and let \( n \geq 2 \). Consider the following proposition:

\textbf{Proposition.} For all integers \( a \) and \( b \), if \( a + b \not\equiv 0 \pmod{n} \), then \( a \not\equiv 0 \pmod{n} \) or \( b \not\equiv 0 \pmod{n} \).

\begin{enumerate}[label=(\alph*)]
    \item Write the contrapositive of the proposition.
    \begin{enumerate}[label=(\alph*)]
        \item For all integers \( a \) and \( b \), if \( a \equiv 0 \pmod{n} \) and \( b \equiv 0 \pmod{n} \), then \( a + b \equiv 0 \pmod{n} \).
    
    \end{enumerate}
    
    \item Construct a know-show table to outline the proof of the proposition (using the contrapositive). You do not need to write a formal proof.

    \begin{center}
        \begin{tabular}{|p{.1\textwidth}|p{.5\textwidth}|p{.3\textwidth}|}
        \hline
        \textbf{Step} & \textbf{Know} & \textbf{Reason} \\
        \hline
            P1 & \( a \equiv 0 \pmod{n} \) and \( b \equiv 0 \pmod{n} \) & Hypothesis \\
        \hline
            P2 & For some integers \( k \) and \( l \), \( a = kn \) and \( b = ln \) & Definition of Congruence \\
        \hline
            P3 & \( a + b = kn + ln = n(k + l) \) & Substitution \\
        \hline
            P4 & \( k + l \in \mathbb{Z} \) & \( \mathbb{Z} \) is closed under multiplication \\
        \hline
            P5 & \( \exists m \in \mathbb{Z} \) such that \( m = k + l \) and \( a + b = m \cdot n \) & Substitute \( m = k + l \) \\
        \hline
            Q1 & \( a + b \equiv 0 \pmod{n} \) & Definition of Congruence \\
        \hline
    \end{tabular}
\end{center}
    
\end{enumerate}
\end{problem}

\bigskip

\begin{problem}
Is the following conjecture true or false? If the conjecture is true, write a formal proof. If the conjecture is false, write a formal counterexample argument. (Be sure to aim for “practitioner” on the writing rubrics).

\textbf{Conjecture.} For all integers \( a \) and \( b \), \( a + b \not\equiv 0 \pmod{n} \) if and only if \( a \not\equiv 0 \pmod{n} \) or \( b \not\equiv 0 \pmod{n} \).

\begin{enumerate}[label=(\alph*)]
    \item The statement, "For all integers \( a \) and \( b \), \( a + b \not\equiv 0 \pmod{n} \) if and only if \( a \not\equiv 0 \pmod{n} \) or \( b \not\equiv 0 \pmod{n} \)," is false. One counterexample is when \( n = 5 \), \( a = 1 \), and \( b = 4 \). In this case, we see the hypothesis is false because \( a + b = 1 + 4 = 5 \equiv 0 \pmod{5} \). However, the conclusion is true because \( a \not\equiv 0 \pmod{5} \) and \( b \not\equiv 0 \pmod{5} \). Since the hypothesis is false and the conclusion is true, the statement is false.
    
\end{enumerate}

\end{problem}

\bigskip

\begin{problem}
Practice using previous results.
\begin{enumerate}[label=(\alph*)]
    \item Write the contrapositive of the lemma below:    
    
    \textbf{Lemma.} If \( n \) is an odd integer, then 4 does not divide \( n \).
        \begin{enumerate}[label=(\alph*)]
            \item If 4 divides \( n \), then \( n \) is an even integer.
        
        \end{enumerate}
    \item Write a formal proof of the lemma.
        \begin{enumerate}[label=(\alph*)]
            \item To prove the lemma by contrapositive, we assume that 4 divides \( n \). By the definition of divisibility, there exists an integer \( k \) such that \( n = 4k \). Using algebra, we get \( n = 2(2k) \). Since \( 2k \) is an integer, there exists an integer \( m \) such that \( m = 2k \). Then we can express \( n \) as:
            \[
            n = 2 \times (2k) = 2m.
            \]
            This shows that \( n \) is divisible by 2, which means \( n \) is an even integer.
            
        \end{enumerate}
    
    \item Consider the following proposition:
    
    \textbf{Proposition.} For all integers \( a \) and \( b \), if \( a \) is even and \( b \) is odd, then 4 does not divide \( a^2 + b^2 \).
    
    Use previously proved results to construct a know-show table that outlines a proof of the proposition.

    \begin{center}
        \begin{tabular}{|p{.1\textwidth}|p{.5\textwidth}|p{.3\textwidth}|}
        \hline
        \textbf{Step} & \textbf{Know} & \textbf{Reason} \\
        \hline
            P1 & \( a \) is even and \( b \) is odd & Hypothesis \\
        \hline
            P2 & There exists \( k \in \mathbb{Z} \) such that \( a = 2k \) & Definition of even \\
        \hline
            P3 & There exists \( m \in \mathbb{Z} \) such that \( b = 2m + 1 \) & Definition of odd \\
        \hline
            P4 & Compute \( a^2 = (2k)^2 = 4k^2 \) & Algebra \\
        \hline
            P5 & Compute \( b^2 = (2m+1)^2 = 4m^2 + 4m + 1 \) & Algebra \\
        \hline
            P6 & Sum \( a^2 + b^2 = 4k^2 + 4m^2 + 4m + 1 = 4(k^2 + m^2 + m) + 1 \) & Substitution \\
        \hline
            P7 & Let \( x = k^2 + m^2 + m \), which is an integer since \( k, m \in \mathbb{Z} \) & \( \mathbb{Z} \) is closed under multiplication \\
        \hline
            P8 & \( a^2 + b^2 = 4x + 1 \) & Substitution \\
        \hline
            P9 & \( a^2 + b^2 \equiv 1 \pmod{4} \) & Definition of congruence \\
        \hline
            P10 & Since \( a^2 + b^2 \equiv 1 \pmod{4} \), \( a^2 + b^2 \) is an odd integer & \\
        \hline
            P11 & Previously proven lemma: If \( n \) is odd, then 4 does not divide \( n \) &  \\
        \hline
            P12 & Since \( a^2 + b^2 \) is odd, 4 does not divide \( a^2 + b^2 \) & By the lemma \\
        \hline
        \textbf{Step} & \textbf{Show} & \textbf{Reason} \\
        \hline
\end{tabular}

    \end{center}
    
\end{enumerate}
\end{problem}

\bigskip

\begin{problem}
Use the proposition and proof on the next page to do the following:
\begin{enumerate}[label=(\alph*)]
    \item Clearly identify the assumptions in the proof.
    \begin{enumerate}[label=(\alph*)]
        \item The assumptions of the proof are:
        \begin{itemize}
            \item \( x \) is a rational number (\( x \in \mathbb{Q} \))
            \item \( x \neq 0 \)
            \item \( y \) is an irrational number (\( y \notin \mathbb{Q} \))
            \item The product \( x \cdot y \) is a rational number (\( x \cdot y \in \mathbb{Q} \))
        \end{itemize}
    \end{enumerate}
    
    \item Clearly identify the contradictory statement.
    \begin{enumerate}[label=(\alph*)]
        \item The contradictory statement is that \( y \) is both rational and irrational. Specifically, the proof concludes that \( y \) must be rational based on the assumptions, which contradicts the initial assumption that \( y \) is irrational.
    \end{enumerate}
    
    \item In 2 - 3 sentences, summarize the main idea of the proof in a way that makes sense to you.
    \begin{enumerate}[label=(\alph*)]
        \item The main idea of the proof is to assume that multiplying a nonzero rational number \( x \) by an irrational number \( y \) yields a rational result. By manipulating this assumption, it deduces that \( y \) must be rational, contradicting the original assumption that \( y \) is irrational. This contradiction implies that the product \( x \cdot y \) cannot be rational, thus proving it must be irrational.
    \end{enumerate}
    
\end{enumerate}

\end{problem}

\bigskip

\hrule

\bigskip
\bigskip

\textbf{Proposition and formal proof for Problem 4}

\textbf{Proposition:} For all real numbers \( x \) and \( y \), if \( x \) is rational and \( x \neq 0 \) and \( y \) is irrational, then \( x \cdot y \) is irrational.

\textbf{Proof.} We will use a proof by contradiction. So we assume that there exist real numbers \( x \) and \( y \) such that \( x \) is rational, \( x \neq 0 \), \( y \) is irrational, and \( x \cdot y \) is rational. Since \( x \neq 0 \), we can divide by \( x \). Since the rational numbers are closed under division by nonzero rational numbers, we know that \( \frac{1}{x} \in \mathbb{Q} \). We now know that \( x \cdot y \) and \( \frac{1}{x} \) are rational numbers, and since the rational numbers are closed under multiplication, we conclude that
\[
\frac{1}{x} \cdot (x \cdot y) \in \mathbb{Q}.
\]
However, \( \frac{1}{x} \cdot (x \cdot y) = y \) and hence, \( y \) must be a rational number. Since a real number cannot be both rational and irrational, this is a contradiction to the assumption that \( y \) is irrational. We have therefore proved that for all real numbers \( x \) and \( y \), if \( x \) is rational and \( x \neq 0 \) and \( y \) is irrational, then \( x \cdot y \) is irrational.

%-------------------------------------------------------------------
%Do not delete \end{document}

\end{document}

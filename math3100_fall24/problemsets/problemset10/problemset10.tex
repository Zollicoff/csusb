% Beginning of preamble 
% Specify the type of document you want (usually article) and the font size
\documentclass[12pt]{article}  

% Packages to load which give you useful commands
\usepackage{amssymb, amsmath, amsthm, amsfonts}

% Some other packages that may be useful
\usepackage{euscript}
\usepackage{graphicx}
\usepackage[dvipsnames]{xcolor}

% Allow for URLs and hyperlinks
\usepackage{url}
\usepackage{hyperref}
\hypersetup{colorlinks=true, urlcolor=RoyalBlue, citecolor=RedViolet}

% Set the margins and other page formatting
\usepackage{geometry}
\geometry{left=1in, right=1in, top=1in, bottom=1in}
\usepackage{enumitem}
\usepackage{setspace} 
\setlength\parindent{0pt}
\linespread{1.2}
%\pagestyle{empty}  %This style will suppress page numbering

% Define a few custom environments with numbering
\newtheorem{problem}{Problem} 
\newtheorem*{theorem}{Theorem}
\newtheorem*{conjecture}{Conjecture}
\theoremstyle{definition} 
\newtheorem*{definition}{Definition}
\newtheorem*{answer}{Answer}
\newtheorem*{example}{Example}

% Set the font to Helvetica
\usepackage{helvet}
\renewcommand{\familydefault}{\sfdefault}

% Custom notation shortcut commands (currently unused)
\newcommand{\Z}{\mathbb{Z}}
% \newcommand{\R}{\mathbb{R}}
% \newcommand{\Q}{\mathbb{Q}}
% \newcommand{\N}{\mathbb{N}}
% \newcommand{\eps}{\varepsilon}
% \newcommand{\solution}{\textcolor{PineGreen}{Solution:\newline}}

%%%%% End of preamble %%%%%
%--------------------------------------------------------------------------

\begin{document}

Math 3100 \hfill Zachary Hampton

Problem Set 10 \hfill Due Date: 11-29-2024

\bigskip

\section*{Problem 1}

\begin{problem}
1. Use roster notation to explicitly list the elements in each of the following sets. When applicable, include any supporting work.

\begin{enumerate}[label=(\alph*)]
    \item $\{\, a \in \mathbb{N} \mid 3a - 4 \leq 17 \,\}$
    \item $\{\, x \in \mathbb{R} \mid 2x^2 + 3x - 2 = 0 \,\}$
    \item $\{\, p \in \mathbb{Z} \mid -4 < \sqrt{p} < 4 \,\}$
    \item $\{\, n \in \mathbb{Z} \mid n \equiv 1 \pmod{4} \,\}$
    \item $\{\, y \in \mathbb{Q} \mid \left| y - \dfrac{1}{3} \right| = \dfrac{8}{3} \,\}$
\end{enumerate}
\end{problem}

\textbf{Solution:}

\begin{enumerate}[label=(\alph*)]
    \item $\{\, a \in \mathbb{N} \mid 3a - 4 \leq 17 \,\}$

    \textit{Work:} Solve the inequality for $a$:
    \[
    3a - 4 \leq 17 \implies 3a \leq 21 \implies a \leq 7.
    \]
    Since $a \in \mathbb{N}$, the set is:
    \[
    \{1, 2, 3, 4, 5, 6, 7\}.
    \]

    \item $\{\, x \in \mathbb{R} \mid 2x^2 + 3x - 2 = 0 \,\}$

    \textit{Work:} Factor the quadratic equation:
    \[
    2x^2 + 3x - 2 = 0.
    \]
    Look for factors of $2x^2$ and $-2$ that sum to $3x$:
    \[
    (2x - 1)(x + 2) = 2x^2 + 4x - x -2 = 2x^2 + 3x -2.
    \]
    Set each factor to zero:
    \[
    2x - 1 = 0 \implies x = \dfrac{1}{2}, \quad x + 2 = 0 \implies x = -2.
    \]
    Therefore, the set is:
    \[
    \left\{ -2,\ \dfrac{1}{2} \right\}.
    \]

    \item $\{\, p \in \mathbb{Z} \mid -4 < \sqrt{p} < 4 \,\}$

    \textit{Work:} Since $\sqrt{p}$ is real, $p \geq 0$. Then:
    \[
    0 \leq \sqrt{p} < 4 \implies 0 \leq p < 16.
    \]
    Therefore, $p$ can be any integer from $0$ to $15$. The set is:
    \[
    \{0, 1, 2, 3, 4, 5, 6, 7, 8, 9, 10, 11, 12, 13, 14, 15\}.
    \]

    \item $\{\, n \in \mathbb{Z} \mid n \equiv 1 \pmod{4} \,\}$

    \textit{Work:} This set includes all integers that leave a remainder of $1$ when divided by $4$:
    \[
    \{\ldots, -7, -3, 1, 5, 9, 13, 17, \ldots\}.
    \]

    \item $\{\, y \in \mathbb{Q} \mid \left| y - \dfrac{1}{3} \right| = \dfrac{8}{3} \,\}$

    \textit{Work:} Solve for $y$:
    \[
    \left| y - \dfrac{1}{3} \right| = \dfrac{8}{3} \implies y - \dfrac{1}{3} = \pm \dfrac{8}{3}.
    \]
    Thus:
    \[
    y = \dfrac{1}{3} + \dfrac{8}{3} = \dfrac{9}{3} = 3, \quad y = \dfrac{1}{3} - \dfrac{8}{3} = -\dfrac{7}{3}.
    \]
    Therefore, the set is:
    \[
    \left\{ -\dfrac{7}{3},\ 3 \right\}.
    \]

\end{enumerate}

\newpage

\section*{Problem 2}

\begin{problem}
2. Let $A$ and $B$ be subsets of a universal set $U$, and let $x \in U$. Write useful negations of the definitions of set intersection, set union, and set difference by completing the sentences below:

\begin{enumerate}[label=(\alph*)]
    \item $x \notin A \cap B$ provided that \underline{\hspace{4cm}}.
    \item $x \notin A \cup B$ provided that \underline{\hspace{4cm}}.
    \item $x \notin A - B$ provided that \underline{\hspace{4cm}}.
\end{enumerate}
\end{problem}

\textbf{Solution:}

\begin{enumerate}[label=(\alph*)]
    \item $x \notin A \cap B$ provided that \textbf{$x \notin A$ or $x \notin B$}.

    \item $x \notin A \cup B$ provided that \textbf{$x \notin A$ and $x \notin B$}.

    \item $x \notin A - B$ provided that \textbf{$x \notin A$ or $x \in B$}.
\end{enumerate}

\newpage

\section*{Problem 3}

\begin{problem}
3. Let $U = \mathbb{N}$ and let:

\begin{align*}
    A &= \{\, x \in \mathbb{N} \mid x \geq 7 \,\}, \\
    B &= \{\, x \in \mathbb{N} \mid x \text{ is odd} \,\}, \\
    C &= \{\, x \in \mathbb{N} \mid x \equiv 0 \pmod{3} \,\}, \\
    D &= \{\, x \in \mathbb{N} \mid x \text{ is even} \,\}.
\end{align*}

Use the roster method to list all of the elements in each of the following sets. When applicable, show your work by finding the sets in parentheses first:

\begin{enumerate}[label=(\alph*)]
    \item $(A \cup B)^\complement$
    \item $A^\complement \cap B^\complement$
    \item $(A \cup B) \cap C$
    \item $B \cap D$
    \item $(B \cap D)^\complement$
    \item $B - D$
\end{enumerate}
\end{problem}

\textbf{Solution:}

\begin{enumerate}[label=(\alph*)]
    \item $(A \cup B)^\complement$

    \textit{Work:} First, find $A \cup B$:
    \[
    A = \{7, 8, 9, 10, 11, \ldots\}, \quad B = \{1, 3, 5, 7, 9, \ldots\}.
    \]
    So:
    \[
    A \cup B = \{1, 3, 5, 7, 8, 9, 10, 11, \ldots\}.
    \]
    Then, the complement with respect to $U = \mathbb{N}$:
    \[
    (A \cup B)^\complement = \{2, 4, 6\}.
    \]

    \item $A^\complement \cap B^\complement$

    \textit{Work:} Find the complements:
    \[
    A^\complement = \{1, 2, 3, 4, 5, 6\}, \quad B^\complement = \{2, 4, 6, 8, 10, \ldots\}.
    \]
    Then:
    \[
    A^\complement \cap B^\complement = \{2, 4, 6\}.
    \]

    \item $(A \cup B) \cap C$

    \textit{Work:} We have:
    \[
    C = \{3, 6, 9, 12, 15, \ldots\}.
    \]
    Then:
    \[
    (A \cup B) \cap C = \{3, 9, 12, 15, 18, \ldots\}.
    \]
    Listing the elements:
    \[
    \{3, 9, 12, 15, 18, 21, 24, \ldots\}.
    \]

    \item $B \cap D$

    \textit{Work:} Since $B$ contains odd numbers and $D$ contains even numbers:
    \[
    B \cap D = \emptyset.
    \]

    \item $(B \cap D)^\complement$

    \textit{Work:} Since $B \cap D = \emptyset$, its complement is:
    \[
    (B \cap D)^\complement = U = \mathbb{N}.
    \]

    \item $B - D$

    \textit{Work:} Since $B$ contains odd numbers and $D$ contains even numbers:
    \[
    B - D = B.
    \]
    So:
    \[
    B - D = \{1, 3, 5, 7, 9, 11, \ldots\}.
    \]

\end{enumerate}

\newpage

\section*{Problem 4}

\begin{problem}
4. Let:

\begin{align*}
    A &= \{1, 2\}, \\
    B &= \{a, b, c, d\}, \\
    C &= \{1, a, b\}.
\end{align*}

Use the roster method to list all of the elements in each of the following sets. When applicable, show your work by finding the sets in parentheses first:

\begin{enumerate}[label=(\alph*)]
    \item $A \times B$
    \item $B \times A$
    \item $A \times (B \cap C)$
    \item $(A \times B) \cap (A \times C)$
\end{enumerate}
\end{problem}

\textbf{Solution:}

\begin{enumerate}[label=(\alph*)]
    \item $A \times B$

    \textit{Work:} Compute the Cartesian product:
    \[
    A \times B = \{ (1, a),\ (1, b),\ (1, c),\ (1, d),\ (2, a),\ (2, b),\ (2, c),\ (2, d) \}.
    \]

    \item $B \times A$

    \textit{Work:} Compute the Cartesian product:
    \[
    B \times A = \{ (a, 1),\ (a, 2),\ (b, 1),\ (b, 2),\ (c, 1),\ (c, 2),\ (d, 1),\ (d, 2) \}.
    \]

    \item $A \times (B \cap C)$

    \textit{Work:} First, find $B \cap C$:
    \[
    B \cap C = \{a, b\}.
    \]
    Then:
    \[
    A \times (B \cap C) = \{ (1, a),\ (1, b),\ (2, a),\ (2, b) \}.
    \]

    \item $(A \times B) \cap (A \times C)$

    \textit{Work:} Compute $A \times C$:
    \[
    A \times C = \{ (1, 1),\ (1, a),\ (1, b),\ (2, 1),\ (2, a),\ (2, b) \}.
    \]
    Then:
    \[
    (A \times B) \cap (A \times C) = \{ (1, a),\ (1, b),\ (2, a),\ (2, b) \}.
    \]

\end{enumerate}

\newpage

\end{document}

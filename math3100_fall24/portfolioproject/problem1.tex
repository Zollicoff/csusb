%  DO NOT touch any of the code that appears in the preamble (i.e before \begin{document}). Removing packages could break things you're trying to do. Adding packages usually isn't a big deal, but some packages don't play well with each other. Keep that in mind.
%--------------------------------------------------------------------------

% Beginning of preamble 
% Specify the type of document you want (usually article) and the font size
\documentclass[12pt]{article}  

% Packages to load which give you useful commands
\usepackage{amssymb, amsmath, amsthm, amsfonts}

% Some other packages that I find useful
\usepackage{euscript}
\usepackage{graphicx}
\usepackage[dvipsnames]{xcolor}
\usepackage{lineno}

% allow for urls and hyperlinks
\usepackage{url}
\usepackage{hyperref}
\hypersetup{colorlinks=true, urlcolor=RoyalBlue, citecolor=RedViolet}

% Set the margins other page formatting
\usepackage{geometry}
    \geometry{left=1in, right=1in, top=1in, bottom=1in}
\usepackage{enumitem}
\usepackage{setspace} 
\setlength\parindent{0pt}
\linespread{1.2}
\linenumbers

% Define a few theorem-type environments
\newtheorem*{theorem}{Theorem}
\newtheorem*{prop}{Proposition}
\newtheorem*{cor}{Corollary}
\newtheorem*{lemma}{Lemma}
\newtheorem*{problem}{Problem}

\theoremstyle{definition} 
\newtheorem*{definition}{Definition}
\newtheorem*{example}{Example}


% Make the font Sans Serif
\renewcommand{\familydefault}{\sfdefault}

% Some notation shortcut commands
\newcommand{\R}{\mathbb{R}} %the reals
\newcommand{\Q}{\mathbb{Q}} %the rationals
\newcommand{\N}{\mathbb{N}} %the natural numbers
\newcommand{\Z}{\mathbb{Z}} %the integers
\newcommand{\eps}{\varepsilon} %a cooler epsilon



%%%%% End of preamble %%%%%
%--------------------------------------------------------------------------

\begin{document}

%%%%% Begin body  %%%%%
\section{Portfolio Problem 1}

\subsection{Examples and Definitions}
\textcolor{Red}{Include your examples/exploratory work and any relevant definitions the reader may need. Your examples must be typed formally using complete sentences. Delete these directions and the textcolor command before submitting your work.}





\subsection{Know-Show Table}
%%%%% Include your know-show table below. You will need to add more rows for your particular problem. %%%%%
\begin{center}
    \begin{tabular}{|p{.1\textwidth}|p{.5\textwidth}|p{.3\textwidth}|}
    \hline
    \textbf{Step} & \textbf{Know} & \textbf{Reason} \\
    \hline
         &  & \\
    \hline
         &  & \\
    \hline
         &  & \\
    \hline
     \textbf{Step} & \textbf{Show} & \textbf{Reason} \\
    \hline
    \end{tabular}
\end{center}

%%%%% Include a second know-show table if your problem contains two propositions %%%%%




%%%%% Include your first attempt at a formal proof. %%%%%
\subsection{First Draft}


\begin{theorem}
Theorem statement goes here.
\end{theorem}



\begin{proof}
Proof of theorem goes here. 
\end{proof}

%%%%% Include another theorem and proof environment below if your problem contains two propositions %%%%%




%%%%% Bibliography %%%%%
\vspace{1em}
\begin{thebibliography}{9}
\bibitem{Sund21} Sundstrom, T., \emph{Mathematical Reasoning: Writing and Proof}, Version 3,  Creative Commons, 2020.


\bibitem{Whatever21} Include any other references: images from the internet,
definition or theorem from a book other than our class text, etc. Change the name in curly brackets to an appropriate reference name. If you have no other references, delete this item.
\end{thebibliography}


%%%%% End body - do not edit after this line %%%%%
%%%%% Do not delete \end{document} %%%%%

 \end{document}


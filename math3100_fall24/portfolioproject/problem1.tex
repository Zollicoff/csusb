%  DO NOT touch any of the code that appears in the preamble (i.e before \begin{document}). Removing packages could break things you're trying to do. Adding packages usually isn't a big deal, but some packages don't play well with each other. Keep that in mind.
%--------------------------------------------------------------------------

% Beginning of preamble 
% Specify the type of document you want (usually article) and the font size
\documentclass[12pt]{article}  

% Packages to load which give you useful commands
\usepackage{amssymb, amsmath, amsthm, amsfonts}

% Some other packages that I find useful
\usepackage{euscript}
\usepackage{graphicx}
\usepackage[dvipsnames]{xcolor}
\usepackage{lineno}

% allow for urls and hyperlinks
\usepackage{url}
\usepackage{hyperref}
\hypersetup{colorlinks=true, urlcolor=RoyalBlue, citecolor=RedViolet}

% Set the margins other page formatting
\usepackage{geometry}
\geometry{left=1in, right=1in, top=1in, bottom=1in}
\usepackage{enumitem}
\usepackage{setspace} 
\setlength\parindent{0pt}
\linespread{1.2}
\linenumbers

% Define a few theorem-type environments
\newtheorem*{theorem}{Theorem}
\newtheorem*{prop}{Proposition}
\newtheorem*{cor}{Corollary}
\newtheorem*{lemma}{Lemma}
\newtheorem*{problem}{Problem}
\theoremstyle{definition} 
\newtheorem*{definition}{Definition}
\newtheorem*{example}{Example}

% Make the font Sans Serif
\renewcommand{\familydefault}{\sfdefault}

% Some notation shortcut commands
\newcommand{\R}{\mathbb{R}} %the reals
\newcommand{\Q}{\mathbb{Q}} %the rationals
\newcommand{\N}{\mathbb{N}} %the natural numbers
\newcommand{\Z}{\mathbb{Z}} %the integers
\newcommand{\eps}{\varepsilon} %a cooler epsilon

%%%%% End of preamble %%%%%
%--------------------------------------------------------------------------

\begin{document}

%%%%% Begin body  %%%%%
\section{Portfolio Problem 1 - Divisibility by 4}

\subsection{Examples and Definitions}

\textbf{Problem to prove:} Construct both a know-show table and a first attempt at a formal proof of the following biconditional statement.

\begin{prop}
    For each integer \( n \), \( n \) is even if and only if 4 divides \( n^2 \).
\end{prop}

\textbf{Exploratory work/Examples:}
\begin{itemize}
    \item Find four even integers \( a \). At least one integer must be negative. Square each integer \( a \) and determine if 4 divides \( a^2 \).
    \begin{itemize}
        \item If we set \( a = -2 \), then \( (-2)^2 = 4 \), and \( 4 \mid 4 \).
        \item If we set \( a = 0 \), then \( 0^2 = 0 \), and \( 4 \mid 0 \).
        \item If we set \( a = 2 \), then \( 2^2 = 4 \), and \( 4 \mid 4 \).
        \item If we set \( a = 4 \), then \( 4^2 = 16 \), and \( 4 \mid 16 \).
    \end{itemize}
    
    \item Find four perfect squares, \( b \), such that \( 4 \mid b \). Then find the integer \( c \) such that \( b = c^2 \). Determine if \( c \) is even or odd.
    \begin{itemize}
        \item If we set \( b = 0 \), then \( 4 \mid 0 \). Using \( b = c^2 \), we get \( 0 = 0^2 \), so \( c = 0 \), which is even.
        \item If we set \( b = 4 \), then \( 4 \mid 4 \). Using \( b = c^2 \), we get \( 4 = 2^2 \), so \( c = 2 \), which is even.
        \item If we set \( b = 16 \), then \( 4 \mid 16 \). Using \( b = c^2 \), we get \( 16 = 4^2 \), so \( c = 4 \), which is even.
        \item If we set \( b = 144 \), then \( 4 \mid 144 \). Using \( b = c^2 \), we get \( 144 = 12^2 \), so \( c = 12 \), which is even.
    \end{itemize}
\end{itemize}

\subsection[]{Know-Show Table: If \( n \) is even, then \( 4 \mid n^2 \) (Forward)}
%%%%% This table handles the forward direction: If \( n \) is even, then \( 4 \mid n^2 \). %%%%%
\begin{center}
    \begin{tabular}{|p{.1\textwidth}|p{.5\textwidth}|p{.3\textwidth}|}
    \hline
    \textbf{Step} & \textbf{Know} & \textbf{Reason} \\
    \hline
        P1 & \( n \) is even & Hypothesis \\
    \hline
        P2 & \( \exists k \in \mathbb{Z} \text{ s.t. } n = 2k \) & Definition of even \\
    \hline
        P3 & \( n^2 = (2k)^2 = 4k^2 \) & Substitution \\
    \hline
        P4 & \( 4k^2 = 2(2k^2) \) & Factoring out \( 2 \) \\
    \hline
        P5 & \( 2k^2 \in \mathbb{Z} \) & \( \mathbb{Z}\) closed under multiplication \\
    \hline
        P6 & \( n^2 = 2q \) where \( q \in \mathbb{Z} \) & Set \( q = k^2 \) \\
    \hline
        Q1 & 4 divides \( n^2 \) & Defn divides \\
    \hline
    \textbf{Step} & \textbf{Show} & \textbf{Reason} \\
    \hline
    \end{tabular}
\end{center}

\subsection[]{Know-Show Table: If \( 4 \mid n^2 \), then \( n \) is even (Reverse)}
%%%%% This table handles the reverse direction: If \( 4 \mid n^2 \), then \( n \) is even. %%%%%
\begin{center}
    \begin{tabular}{|p{.1\textwidth}|p{.55\textwidth}|p{.25\textwidth}|}
    \hline
    \textbf{Step} & \textbf{Know} & \textbf{Reason} \\
    \hline
        P1 & \( 4 \mid n^2 \) & Hypothesis \\
    \hline
        P2 & \( n^2 = 4m \) for some \( m \in \mathbb{Z} \) & Definition of divisibility \\
    \hline
        P3 & \( n^2 \equiv 0 \mod 4 \) & Definition of congruence \\
    \hline
        P4 & Assume \( n \) is odd & For contradiction \\
    \hline
        P5 & \( n = 2k + 1 \) for some \( k \in \mathbb{Z} \) & Definition of odd \\
    \hline
        P6 & \( n^2 = (2k + 1)^2 = 4k^2 + 4k + 1 \) & Expand and simplify \( n^2 \) \\
    \hline
        P7 & \( n^2 = 4(k^2 + k) + 1 \) & Factor out 4 \\
    \hline
        P8 & \( k^2 + k \in \mathbb{Z} \) & \( \mathbb{Z}\) closed under multiplication \\
    \hline
        P9 & \( n^2 \equiv 1 \mod 4 \) & \( 4(k^2 + k) \equiv 0 \mod 4 \) \\
    \hline
        P10 & Contradicts \( n^2 \equiv 0 \mod 4 \) & From P2, since \( n^2 = 4m \) \\
    \hline
        Q1 & \( n \) is even & Definition of even \\
    \hline
    \textbf{Step} & \textbf{Show} & \textbf{Reason} \\
    \hline
    \end{tabular}
\end{center}


%%%%% Include a second know-show table if your problem contains two propositions %%%%%


%%%%% Include your first attempt at a formal proof. %%%%%
\subsection{First Draft}


\begin{theorem}
    For each integer \( n \), \( n \) is even if and only if 4 divides \( n^2 \).
\end{theorem}




\begin{proof}
Proofy McProoferson goes here. 
\end{proof}

%%%%% Include another theorem and proof environment below if your problem contains two propositions %%%%%




%%%%% Bibliography %%%%%
\vspace{1em}
\begin{thebibliography}{9}
\bibitem{Sund21} Sundstrom, T., \emph{Mathematical Reasoning: Writing and Proof}, Version 3,  Creative Commons, 2020.


\bibitem{Whatever21} Include any other references: images from the internet,
definition or theorem from a book other than our class text, etc. Change the name in curly brackets to an appropriate reference name. If you have no other references, delete this item.
\end{thebibliography}


%%%%% End body - do not edit after this line %%%%%
%%%%% Do not delete \end{document} %%%%%

 \end{document}


% Beginning of preamble
% Specify the type of document you want (usually article) and the font size
\documentclass[12pt]{article}

% Packages to load which give you useful commands
\usepackage{amssymb, amsmath, amsthm, amsfonts}

% Some other packages that I find useful
\usepackage{euscript}
\usepackage{graphicx}
\usepackage[dvipsnames]{xcolor}
\usepackage{lineno}

% allow for urls and hyperlinks
\usepackage{url}
\usepackage{hyperref}
\hypersetup{colorlinks=true, urlcolor=RoyalBlue, citecolor=RedViolet}

% Set the margins other page formatting
\usepackage{geometry}
\geometry{left=1in, right=1in, top=1in, bottom=1in}
\usepackage{enumitem}
\usepackage{setspace}
\setlength\parindent{0pt}
\linespread{1.2}
\linenumbers

% Define a few theorem-type environments
\newtheorem*{theorem}{Theorem}
\newtheorem*{prop}{Proposition}
\newtheorem*{cor}{Corollary}
\newtheorem*{lemma}{Lemma}
\newtheorem*{problem}{Problem}
\theoremstyle{definition}
\newtheorem*{definition}{Definition}
\newtheorem*{example}{Example}

% Make the font Sans Serif
\renewcommand{\familydefault}{\sfdefault}

% Some notation shortcut commands
\newcommand{\R}{\mathbb{R}} %the reals
\newcommand{\Q}{\mathbb{Q}} %the rationals
\newcommand{\N}{\mathbb{N}} %the natural numbers
\newcommand{\Z}{\mathbb{Z}} %the integers
\newcommand{\eps}{\varepsilon} %a cooler epsilon

% End of preamble
%--------------------------------------------------------------------------

\begin{document}

\section{Portfolio Problem 2 - C. Squares and Cubes}

\subsection{Examples and Definitions}

\textbf{Problem to prove:} Construct both a know-show table and a first attempt at a formal proof of the following proposition.

\begin{prop}
    Let \( a, b \in \N \). If \( a^2 = b^3 \) and \( a \) is even, then \( 4 \mid a \).
\end{prop}

\subsection{Exploratory Work/Examples}

\textbf{Strategy:} 
To find four pairs of natural numbers \( (a, b) \) such that \( a^2 = b^3 \) and \( a \) is even, I modified my approach:
\begin{itemize}
    \item Filtered \( a \) to include only even numbers.
    \item Verified whether \( \sqrt[3]{a^2} \) is an integer and that \( b^3 = a^2 \).
    \item Added additional columns to check if \( a \) is even and ensure that both \( a \) and \( b \) satisfy the conditions for being natural numbers.
\end{itemize}

\textbf{Excel Formulas Used:}
\begin{itemize}
    \item Column A: Even natural numbers \( a \).
        \begin{itemize}
            \item Formula: \texttt{=even a value}
        \end{itemize}
    \item Column B: Squares of \( a \) (\( a^2 \)).
        \begin{itemize}
            \item Formula: \texttt{=POWER(A2, 2)}
        \end{itemize}
    \item Column C: Cube root of \( a^2 \) (\( \sqrt[3]{a^2} \)).
        \begin{itemize}
            \item Formula: \texttt{=POWER(B2, 1/3)}
        \end{itemize}
    \item Column D: Verification column for \( b^3 \) (\( b^3 \in \N \)).
        \begin{itemize}
            \item Formula: \texttt{=IF(AND(C2>0, C2=INT(C2)), C2*C2*C2 \& "\(\in\mathbb{N}\)", "b\(\notin\)\mathbb{N}")}
        \end{itemize}
    \item Column E: Even or Odd verification for \( a \).
        \begin{itemize}
            \item Formula: \texttt{=IF(A2=INT(A2), IF(MOD(A2,2)=0, "Even", "Odd"), "Not an Integer")}
        \end{itemize}
    \item Column F: Final check for Even and Natural.
        \begin{itemize}
            \item Formula: \texttt{=IF(AND(A2>0, A2=INT(A2), MOD(A2,2)=0, C2>0, C2=INT(C2)), "Natural and Even", "False")}
        \end{itemize}
\end{itemize}

\textbf{Results:} The four pairs satisfying \( a^2 = b^3 \), with \( a \) even, are:
\begin{center}
    \begin{tabular}{|c|c|c|c|}
    \hline
    \( a \) & \( a^2 \) & \( b \) & \( b^3 \) \\
    \hline
    8 & 64 & 4 & 64 \\
    64 & 4096 & 16 & 4096 \\
    216 & 46656 & 36 & 46656 \\
    512 & 262144 & 64 & 262144 \\
    \hline
    \end{tabular}
\end{center}

\textbf{Spreadsheet Screenshot:}
\begin{center}
    \includegraphics[width=\textwidth]{Portfolio Problem 2 Spreadsheet.png}


\end{center}

\newpage

\subsection{Know-Show Table (\( a^2 = b^3 \) and \( a \) even \( \implies 4 \mid a \))}

\begin{center}
    \begin{tabular}{|p{.1\textwidth}|p{.6\textwidth}|p{.2\textwidth}|}
    \hline
    \textbf{Step} & \textbf{Know} & \textbf{Reason} \\
    \hline
        P1 & \( a^2 = b^3 \) & Hypothesis \\
    \hline
        P2 & \( a \) is even (\( a = 2k \)) & Hypothesis \\
    \hline
        P3 & \( (2k)^2 = b^3 \) & Substitution \\
    \hline
        P4 & \( 4k^2 = b^3 \) & Algebra \\
    \hline
        P5 & \( b \) is even (\( b = 2m \)) & Cubes divisible by 4 imply base divisible by 2 \\
    \hline
        P6 & Substituting \( b = 2m \): \( 4k^2 = (2m)^3 \) & Substitution \\
    \hline
        P7 & \( 4k^2 = 8m^3 \) & Algebra \\
    \hline
        P8 & \( k^2 = 2m^3 \) & Divide by 4 \\
    \hline
        P9 & \( k \) is even (\( k = 2n \)) & Squares divisible by 2 imply base divisible by 2 \\
    \hline
        Q1 & \( a = 4n \) & Substituting \( k = 2n \) into \( a = 2k \) \\
    \hline
    \textbf{Step} & \textbf{Show} & \textbf{Reason} \\
    \hline
    \end{tabular}
\end{center}

\newpage

\subsection{First Draft of Proof}

\begin{proof}
    Assume \( a^2 = b^3 \), and \( a \) is even. Then there exists an integer \( k \) such that \( a = 2k \). Substituting into \( a^2 \), we get:
    \[
    a^2 = (2k)^2 = 4k^2
    \]
    By the hypothesis \( a^2 = b^3 \), we know \( b^3 = 4k^2 \). Since \( b^3 \) is divisible by 4, \( b \) must be even. Let \( b = 2m \) for some integer \( m \). Substituting into \( b^3 \), we get:
    \[
    b^3 = (2m)^3 = 8m^3
    \]
    Thus:
    \[
    4k^2 = 8m^3
    \]
    Dividing both sides by 4, we find:
    \[
    k^2 = 2m^3
    \]
    Since \( k^2 \) is even, \( k \) must also be even. Let \( k = 2n \) for some integer \( n \). Substituting into \( a = 2k \), we get:
    \[
    a = 2(2n) = 4n
    \]
    Therefore, \( a \) is divisible by 4.
\end{proof}

\newpage

\subsection{Second Draft of Proof}

We aim to prove the following proposition:

\textbf{Proposition:} Let \( a, b \in \mathbb{N} \). If \( a^2 = b^3 \) and \( a \) is even, then \( 4 \mid a \).

\begin{proof}
    Assume \( a^2 = b^3 \), and \( a \) is even. Since \( a \) is even, there exists an integer \( k \) such that \( a = 2k \). Substituting \( a = 2k \) into \( a^2 \), we get:
    \[
    a^2 = (2k)^2 = 4k^2.
    \]

    By the hypothesis \( a^2 = b^3 \), it follows that:
    \[
    b^3 = 4k^2.
    \]

    Since \( b^3 \) is divisible by \( 4 \), \( b \) must also be even. Let \( b = 2m \) for some integer \( m \). Substituting \( b = 2m \) into \( b^3 \), we have:
    \[
    b^3 = (2m)^3 = 8m^3.
    \]

    Thus, the equation becomes:
    \[
    4k^2 = 8m^3.
    \]

    Dividing both sides by \( 4 \), we find:
    \[
    k^2 = 2m^3.
    \]

    Since \( k^2 \) is even, \( k \) must also be even. Let \( k = 2n \) for some integer \( n \). Recalling that \( a = 2k \), we substitute \( k = 2n \) into the expression for \( a \), and we have:  
    \[
    a = 2(2n) = 4n.
    \]

    Hence, \( a \) is divisible by \( 4 \), which proves the proposition that if \( a^2 = b^3 \) and \( a \) is even, then \( a \) must be divisible by \( 4 \).
\end{proof}
\newpage

\subsection{Reflection}

\begin{itemize}
    \item The examples identified pairs \( (a, b) \) that satisfy \( a^2 = b^3 \). These examples confirmed the hypothesis.
    \item The proof used substitution and divisibility properties to show that \( a \) must be divisible by 4 if \( a^2 = b^3 \) and \( a \) is even.
\end{itemize}

% Bibliography
\vspace{1em}
\begin{thebibliography}{9}
\bibitem{Sund21} Sundstrom, T., \emph{Mathematical Reasoning: Writing and Proof}, Version 3,  Creative Commons, 2020.
\end{thebibliography}

\end{document}

% Beginning of preamble 
\documentclass[12pt]{article}  

% Packages to load which give you useful commands
\usepackage{amssymb, amsmath, amsthm, amsfonts}
\usepackage{euscript}
\usepackage{graphicx}
\usepackage[dvipsnames]{xcolor}

% allow for urls and hyperlinks
\usepackage{url}
\usepackage{hyperref}
\hypersetup{colorlinks=true, urlcolor=RoyalBlue, citecolor=RedViolet}

% Set the margins and other page formatting
\usepackage{geometry}
\geometry{left=1in, right=1in, top=1in, bottom=1in}
\usepackage{enumitem}
\usepackage{setspace} 
\setlength\parindent{0pt}
\linespread{1.2}

% Define a few theorem-type environments with numbering
\newtheorem{problem}{Problem} % Automatically numbers each problem
\theoremstyle{definition} 
\newtheorem*{answer}{Answer}

% Make the font Helvetica
\usepackage{helvet}
\renewcommand{\familydefault}{\sfdefault}

% Some notation shortcut commands
\newcommand{\R}{\mathbb{R}} % the reals
\newcommand{\Q}{\mathbb{Q}} % the rationals
\newcommand{\N}{\mathbb{N}} % the natural numbers
\newcommand{\Z}{\mathbb{Z}} % the integers
\newcommand{\eps}{\varepsilon} % a cooler epsilon
\newcommand{\solution}{\textcolor{PineGreen}{\textbf{Solution:}}\newline}

% Load the listings package for code boxes
\usepackage{listings}

%%%%% End of preamble %%%%%
%--------------------------------------------------------------------------

\begin{document}

\begin{flushright}
CSE 5160 Machine Learning \hfill Giovanni Almaraz, Doug Fry, Zachary Hampton\\
Dr. Haiyan Qiao \hfill 02-12-2025
\end{flushright}

\bigskip

\section*{Group Assignment 1}

\begin{problem}
Describe the null hypotheses to which the p-values given in Table 3.4
correspond. Explain what conclusions you can draw based on these
p-values. Your explanation should be phrased in terms of sales, TV,
radio, and newspaper, rather than in terms of the coefficients of the
linear model.

\solution
\textbf{Null Hypotheses:}
\begin{itemize}
    \item \( H_0 \) for TV: There is no relationship between TV advertising budget and sales \(\beta_1 = 0\).
    \item \( H_0 \) for Radio: There is no relationship between radio advertising budget and sales \(\beta_2 = 0\).
    \item \( H_0 \) for Newspaper: There is no relationship between newspaper advertising budget and sales \(\beta_3 = 0\).
\end{itemize}

\textbf{P-values:}
\begin{itemize}
    \item TV (\( p < 0.0001 \)): The very low p-value indicates strong evidence against the null hypothesis, meaning TV advertising has a significant positive effect on sales.
    \item Radio (\( p < 0.0001 \)): The very low p-value also suggests strong evidence against the null hypothesis, meaning radio advertising significantly impacts sales.
    \item Newspaper (\( p = 0.8599 \)): The high p-value suggests we fail to reject the null hypothesis. This means newspaper advertising does not have a statistically significant impact on sales.
\end{itemize}
\end{problem}

\bigskip

% Ensure the problem numbers continue correctly
\setcounter{problem}{2}

\begin{problem}
Suppose we have a data set with five predictors, 
\[
X_1 = \text{GPA}, \quad X_2 = \text{IQ}, \quad X_3 = \text{Level (1 for College, 0 for High School)},
\]
\[
X_4 = \text{Interaction between GPA and IQ}, \quad X_5 = \text{Interaction between GPA and Level}.
\]
The response is starting salary after graduation (in thousands of dollars). Suppose we use least squares to fit the model and get

\[
\hat{\beta}_0 = 50, \quad \hat{\beta}_1 = 20, \quad \hat{\beta}_2 = 0.07, \quad \hat{\beta}_3 = 35, \quad \hat{\beta}_4 = 0.01, \quad \hat{\beta}_5 = -10.
\]

(a) Which answer is correct, and why?
\begin{enumerate}
    \item[i.] For a fixed value of IQ and GPA, high school graduates earn more, on average, than college graduates.
    \item[ii.] For a fixed value of IQ and GPA, college graduates earn more, on average, than high school graduates.
    \item[iii.] For a fixed value of IQ and GPA, high school graduates earn more, on average, than college graduates provided that the GPA is high enough.
    \item[iv.] For a fixed value of IQ and GPA, college graduates earn more, on average, than high school graduates provided that the GPA is high enough.
\end{enumerate}

\solution
For high school graduates (\(X_3 = 0\)), the salary model simplifies to:
\[
\hat{Y}_{\text{HS}} = 50 + 20 X_1 + 0.07 X_2 + 0.01 X_1 X_2.
\]
For college graduates (\(X_3 = 1\)):
\[
\hat{Y}_{\text{College}} = 50 + 20 X_1 + 0.07 X_2 + 35 + 0.01 X_1 X_2 - 10 X_1.
\]
\[
= 85 + 10 X_1 + 0.07 X_2 + 0.01 X_1 X_2.
\]
Since the coefficient for GPA is larger for high school graduates (\(20\)) than for college graduates (\(10\)), high school graduates may earn more when GPA is sufficiently high. Therefore, option **iii** is correct.

\bigskip

(b) Predict the salary of a college graduate with IQ of 110 and a GPA of 4.0.

\bigskip

\solution
Substituting \(X_1 = 4.0\), \(X_2 = 110\), and \(X_3 = 1\):
\[
\hat{Y} = 85 + (10 \times 4.0) + (0.07 \times 110) + (0.01 \times 4.0 \times 110).
\]
\[
= 85 + 40 + 7.7 + 4.4 = 137.1.
\]
Thus, the predicted salary is \(\$137,100\).

\bigskip

(c) True or false: Since the coefficient for the GPA/IQ interaction term (\(\beta_4 = 0.01\)) is very small, there is very little evidence of an interaction effect.

\bigskip

\solution
A small coefficient suggests a weak interaction effect, but without statistical significance (p-values), we cannot confirm its relevance. The statement is **likely true**, but statistical testing is required to conclude definitively.
\end{problem}

\bigskip

% Ensure the problem numbers continue correctly
\setcounter{problem}{7}

\begin{problem}
This question involves the use of simple linear regression on the \texttt{Auto} data set.

\begin{enumerate}[label=(\alph*)]
    \item Use the \texttt{lm()} function to perform a simple linear regression with \texttt{mpg} as the response and \texttt{horsepower} as the predictor. Use the \texttt{summary()} function to print the results. Comment on the output. For example:
    \begin{enumerate}[label=\roman*.]
        \item Is there a relationship between the predictor and the response?
        \item How strong is the relationship between the predictor and the response?
        \item Is the relationship between the predictor and the response positive or negative?
        \item What is the predicted \texttt{mpg} associated with a \texttt{horsepower} of 98? What are the associated 95\% confidence and prediction intervals?
    \end{enumerate}
    
    % Insert the terminal output as one code box:

    \solution
    
    \begin{lstlisting}[frame=single, caption={Terminal Output for Problem 8(a)}]
> # Load the necessary package and data from ISLR2
> library(ISLR2)
> 
> data(Auto)
> 
> # (a) Fit a simple linear regression model with mpg as the response
> # and horsepower as the predictor, then print a summary of the fit.
> model <- .... [TRUNCATED] 
> 
> summary(model)
> 
> Call:
> lm(formula = mpg ~ horsepower, data = Auto)
> 
> Residuals:
>      Min       1Q   Median       3Q      Max 
> -13.5710  -3.2592  -0.3435   2.7630  16.9240 
> 
> Coefficients:
>              Estimate Std. Error t value Pr(>|t|)    
> (Intercept) 39.935861   0.717499   55.66   <2e-16 ***
> horsepower  -0.157845   0.006446  -24.49   <2e-16 ***
> ---
> Signif. codes:  0 ‘***’ 0.001 ‘**’ 0.01 ‘*’ 0.05 ‘.’ 0.1 ‘ ’ 1
> 
> Residual standard error: 4.906 on 390 degrees of freedom
> Multiple R-squared:  0.6059,	Adjusted R-squared:  0.6049 
> F-statistic: 599.7 on 1 and 390 DF,  p-value: < 2.2e-16
> 
> 
> # (b) Plot the data (mpg vs. horsepower) and add the least squares regression line.
> plot(Auto$horsepower, Auto$mpg,
> +      main = "MPG vs. Horsepo ..." ... [TRUNCATED] 
> 
> abline(model, col = "red", lwd = 2)  # add regression line in red
> 
> # (c) Produce diagnostic plots for the fitted model.
> # Set up a 2x2 plotting layout.
> par(mfrow = c(2, 2))
> 
> plot(model)
> 
> # Reset plotting layout to default.
> par(mfrow = c(1, 1))
    \end{lstlisting}
    
    \item Plot the response and the predictor. Use the \texttt{abline()} function to display the least squares regression line.
    
    \item Use the \texttt{plot()} function to produce diagnostic plots of the least squares regression fit. Comment on any problems you see with the fit.

    \solution
    There doesn't seem to be any issues with the fit.

    \begin{figure}[h]
    \centering
    \includegraphics[width=0.8\textwidth]{Rplot.png} % Replace with actual filename
    \caption{Screenshot for Part B and C}
    \end{figure}

    

\end{enumerate}
\end{problem}

\end{document}

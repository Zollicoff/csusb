\documentclass[a4paper,12pt]{article}
\usepackage{graphicx} % For including images
\usepackage{listings} % For including code
\usepackage{hyperref} % For clickable links (if needed)
\usepackage{geometry}
\usepackage{xcolor} % For colored syntax highlighting
\geometry{left=1in, right=1in, top=1in, bottom=1in}

\lstset{
    frame=single, % Adds a box around the code
    numbers=left, % Adds line numbers
    numberstyle=\tiny, % Styles the line numbers
    basicstyle=\ttfamily\small, % Sets the font
    keywordstyle=\color{blue}, % Colors keywords
    commentstyle=\color{cyan}, % Colors comments
    stringstyle=\color{red}, % Colors strings
    breaklines=true, % Enables line breaking
    backgroundcolor=\color{gray!10} % Light gray background
}

\title{Lab Report\\ CSE4010 Computer Architecture}
\author{}
\date{}

\begin{document}

\maketitle

\noindent \textbf{Name:} \underline{Zachary A. Hampton\hspace{5cm}} \hfill \textbf{Score:} \underline{\hspace{2cm}/10} \\\\
\textbf{Student ID:} \underline{008339494\hspace{6cm}} \hfill \textbf{Due:} \underline{02-09-2025} \\\\
\textbf{Lab:} \underline{Lab 1 - Introduction to Verilog\hspace{9cm}}

\section*{Report}
\begin{itemize}
    \item What is Verilog and what is it used for?
        \begin{itemize}
            \item Verilog is an IEEE-standardized Hardware Description Language (HDL) used for designing and modeling digital electronic systems, particularly in FPGA and ASIC development.
        \end{itemize}
    \item What is a module and a testbench?
    \begin{itemize}
            \item A module is the fundamental building block in Verilog; it defines a circuit's behavior, encapsulating functionality with specified inputs, outputs, internal logic, and potentially other modules.
            \item A testbench is a Verilog module used for simulation, providing stimulus to a design under test (DUT) and verifying its behavior before actual hardware implementation.
        \end{itemize}
    \item A brief but comprehensive description of what you’ve done and what you learned.
    \begin{itemize}
            \item For Part A, I set up the `wireTest` module and its corresponding testbench using the provided sample code. I then compiled and executed the simulation, generating a `.vcd` waveform file for analysis. Using GTKWave, I loaded the `.vcd` file and enabled the necessary variables to visualize their respective waveforms. For Part B, I wrote a similar module with an additional input and slightly different logic operations, resulting in a more complex waveform output.
        \end{itemize}
\end{itemize}

\section*{Source Code for Part A}
\subsection*{wireTest.v}
\begin{lstlisting}[language=Verilog, caption=wireTest.v]
module wireTest (A,B,C);

    input A;
    output B;
    output C;

    assign B = A;
    assign C = !A;

endmodule
\end{lstlisting}

\subsection*{wireTest\_tb.v}
\begin{lstlisting}[language=Verilog, caption=wireTest_tb.v]
`timescale 1ns / 1ns
`include "wireTest.v"

module wireTest_tb;

reg A;
wire B;
wire C;

wireTest uut (A, B, C);

initial begin

   $dumpfile("wireTest_tb.vcd");
   $dumpvars(0, wireTest_tb);

   A = 0;
   #20

   A = 1;
   #20

   A = 0;
   #20

   A = 1;
   #20

   $display("Wire test complete!");

end

endmodule
\end{lstlisting}

\section*{Source Code for Part B}
\subsection*{wireTest2.v}
\begin{lstlisting}[language=Verilog, caption=wireTest2.v]
/*
Instructions from Dr. Ahmed:
Accepts two inputs W and X
Gives two outputs Y and Z
Set Y equal to the negation of X
Set Z equal to the negation of Y
*/

// Define the wireTest2 module
module wireTest2 (W,X,Y,Z);

    // Declare the inputs and outputs
    input W;
    input X;
    output Y;
    output Z;

    // Assign Y to the negation of X
    assign Y = !X;
    // Assign Z to the negation of Y
    assign Z = !Y;

// End the wireTest2 module
endmodule
\end{lstlisting}

\subsection*{wireTest2\_tb.v}
\begin{lstlisting}[language=Verilog, caption=wireTest2_tb.v]
/*
Instructions from Dr. Ahmed:
This file is used to test the wireTest2 module.
The testbench will set W and X to 0 and 1 in various combinations
and check that Y and Z are set correctly.
The testbench will output "Wire test complete!" when the test is done.
*/

// Include the wireTest2 module and timescale of 1ns
`timescale 1ns / 1ns
`include "wireTest2.v"

// Define the wireTest2_tb module
module wireTest2_tb;

// Declare the inputs and outputs
reg W;
reg X;
wire Y;
wire Z;

// Instantiate the wireTest2 module
wireTest2 uut (W, X, Y, Z);

// Define the initial block
initial begin

    // Dump the waveforms to wireTest2_tb.vcd
    $dumpfile("wireTest2_tb.vcd");
    $dumpvars(0, wireTest2_tb);

    // Set W and X to 0
    W = 0;
    X = 0;
    #20

    // Set W to 1
    W = 1;
    #20

    // Set X to 1
    X = 1;
    #20

    // Set W to 0
    W = 0;
    #20

    // Set X to 0
    X = 0;
    #20

    // Print "Wire test complete!" to the console
    $display("Wire test complete!");

// End the initial block
end

// End the wireTest2_tb module
endmodule
\end{lstlisting}

\newpage

\section*{Screenshots for Parts A and B}
\begin{figure}[h]
    \centering
    \includegraphics[width=0.8\textwidth]{lab_1a_screenshot.png} % Replace with actual filename
    \caption{Screenshot for Part A}
\end{figure}

\begin{figure}[h]
    \centering
    \includegraphics[width=0.8\textwidth]{lab_1b_screenshot.png} % Replace with actual filename
    \caption{Screenshot for Part B}
\end{figure}

\end{document}

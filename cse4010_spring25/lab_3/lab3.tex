% Lab 3 Report - Full Subtractor Implementation
\documentclass[12pt]{article}
\usepackage[utf8]{inputenc}
\usepackage{graphicx}
\usepackage{listings}
\usepackage{color}
\usepackage{float}
\usepackage{amsmath}

% Define Verilog syntax highlighting
\definecolor{codegreen}{rgb}{0,0.6,0}
\definecolor{codegray}{rgb}{0.5,0.5,0.5}
\definecolor{codepurple}{rgb}{0.58,0,0.82}
\definecolor{backcolour}{rgb}{0.95,0.95,0.92}

\lstdefinestyle{verilog}{
    backgroundcolor=\color{backcolour},   
    commentstyle=\color{codegreen},
    keywordstyle=\color{blue},
    numberstyle=\tiny\color{codegray},
    stringstyle=\color{codepurple},
    basicstyle=\ttfamily\footnotesize,
    breakatwhitespace=false,         
    breaklines=true,                 
    captionpos=b,                    
    keepspaces=true,                 
    numbers=left,                    
    numbersep=5pt,                  
    showspaces=false,                
    showstringspaces=false,
    showtabs=false,                  
    tabsize=2,
    language=Verilog
}

\title{Lab 3: Full Subtractor Implementation and Testing}
\author{Your Name}
\date{\today}

\begin{document}

\maketitle

\section{Introduction}
This lab focuses on the design, implementation, and testing of a 1-bit full subtractor using Verilog HDL. A full subtractor is a combinational circuit that performs subtraction of three binary digits: minuend (A), subtrahend (B), and borrow-in (Bin). The circuit produces two outputs: difference (diff) and borrow-out (Bout).

\section{Design Approach}
The full subtractor is implemented using a hierarchical design approach with the following components:
\begin{itemize}
    \item Half Subtractor module
    \item Full Subtractor module (using two half subtractors)
\end{itemize}

\subsection{Half Subtractor}
The half subtractor is the basic building block that subtracts two bits and produces a difference and borrow output. The logic equations for the half subtractor are:
\begin{align*}
    \text{diff} &= A \oplus B \text{ (XOR)} \\
    \text{borrow} &= \overline{A} \cdot B \text{ (AND with NOT)}
\end{align*}

\subsection{Full Subtractor}
The full subtractor is constructed using two half subtractors and an OR gate. The first half subtractor subtracts B from A, and the second half subtractor subtracts the borrow-in (Bin) from the difference of the first stage. The final borrow-out is the OR of the borrows from both half subtractors.

\section{Implementation}
\subsection{Verilog Code}
\subsubsection{Half Subtractor Module}
\begin{lstlisting}[style=verilog]
module halfSubtractor(op1, op2, diff, borrow);
    input op1, op2;
    output diff, borrow;
    
    assign diff = op1 ^ op2;
    assign borrow = op1 & !op2;
endmodule
\end{lstlisting}

\subsubsection{Full Subtractor Module}
\begin{lstlisting}[style=verilog]
module fullSubtractor(A, B, Bin, diff, Bout);
    input A, B, Bin;
    output diff, Bout;
    
    wire c, d, e, f;
    
    halfSubtractor u1(A, B, c, d);
    halfSubtractor u2(c, Bin, diff, f);
    assign Bout = f | d;
endmodule
\end{lstlisting}

\section{Testing Strategy}
The testbench implements a comprehensive testing approach by:
\begin{itemize}
    \item Testing all possible input combinations (8 test cases)
    \item Using appropriate time delays between test cases
    \item Generating waveform output for verification
\end{itemize}

\subsection{Test Cases}
The following test cases were implemented:
\begin{center}
\begin{tabular}{|c|c|c|c|c|}
\hline
A & B & Bin & diff & Bout \\
\hline
0 & 0 & 0 & 0 & 0 \\
0 & 0 & 1 & 1 & 1 \\
0 & 1 & 0 & 1 & 1 \\
0 & 1 & 1 & 0 & 1 \\
1 & 0 & 0 & 1 & 0 \\
1 & 0 & 1 & 0 & 0 \\
1 & 1 & 0 & 0 & 0 \\
1 & 1 & 1 & 1 & 1 \\
\hline
\end{tabular}
\end{center}

\section{Results and Analysis}
[This section will be completed after running the simulation and analyzing the results]

\section{Conclusion}
[This section will summarize the lab findings and learning outcomes]

\end{document} 
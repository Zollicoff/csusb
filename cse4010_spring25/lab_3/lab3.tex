% Lab 3 Report - Full Subtractor Implementation
\documentclass[a4paper,12pt]{article}
\usepackage{graphicx}
\usepackage{listings}
\usepackage{hyperref}
\usepackage{geometry}
\usepackage{xcolor}
\usepackage{amsmath}
\geometry{left=1in, right=1in, top=1in, bottom=1in}

\lstset{
    frame=single,
    numbers=left,
    numberstyle=\tiny,
    basicstyle=\ttfamily\small,
    keywordstyle=\color{blue},
    commentstyle=\color{cyan},
    stringstyle=\color{red},
    breaklines=true,
    backgroundcolor=\color{gray!10}
}

\title{Lab Report\\ CSE4010 Computer Architecture}
\author{}
\date{}

\begin{document}

\maketitle

\noindent \textbf{Name:} \underline{Zachary A. Hampton\hspace{5cm}} \hfill \textbf{Score:} \underline{\hspace{2cm}/10} \\
\textbf{Student ID:} \underline{008339494\hspace{6cm}} \hfill \textbf{Due:} \underline{02-23-2025} \\
\textbf{Lab:} \underline{Lab 3 - Full Subtractor Implementation\hspace{6cm}}

\section*{Report}
\begin{itemize}
    \item What is a full subtractor? How does it differ from a half subtractor?
    \begin{itemize}
        \item A full subtractor is a combinational circuit that performs subtraction of three binary digits: minuend (A), subtrahend (B), and borrow-in (Bin), producing difference (diff) and borrow-out (Bout) outputs.
        \item Unlike a half subtractor which only handles two inputs (A and B), a full subtractor can process a borrow from a previous stage, making it suitable for multi-bit subtraction operations.
    \end{itemize}
    
    \item How can a full subtractor be constructed using half subtractors?
    \begin{itemize}
        \item A full subtractor can be built using two half subtractors and an OR gate:
        \begin{itemize}
            \item The first half subtractor subtracts B from A
            \item The second half subtractor subtracts Bin from the difference of the first stage
            \item The final borrow-out is the OR of both half subtractor borrows
        \end{itemize}
    \end{itemize}
\end{itemize}

\newpage

\section*{Source Code}
\subsection*{halfSubtractor.v}
\begin{lstlisting}[language=Verilog]
module halfSubtractor(
    input op1,
    input op2,
    output diff,
    output borrow
);
    
    assign diff = op1 ^ op2;
    assign borrow = !op1 & op2;

endmodule
\end{lstlisting}

\newpage

\subsection*{fullSubtractor.v}
\begin{lstlisting}[language=Verilog]
module fullSubtractor(
    input A,
    input B,
    input Bin,
    output diff,
    output Bout
);
    
    wire diff1, borrow1, borrow2;
    
    halfSubtractor hs1(
        .op1(A),
        .op2(B),
        .diff(diff1),
        .borrow(borrow1)
    );
    
    halfSubtractor hs2(
        .op1(diff1),
        .op2(Bin),
        .diff(diff),
        .borrow(borrow2)
    );
    
    assign Bout = borrow1 | borrow2;

endmodule
\end{lstlisting}

\newpage

\subsection*{fullSubtractor\_tb.v}
\begin{lstlisting}[language=Verilog]
`timescale 1ns/1ns
`include "halfSubtractor.v"
`include "fullSubtractor.v"

module fullSubtractor_tb;
    reg A, B, Bin;
    wire diff, Bout;
    
    fullSubtractor uut(
        .A(A),
        .B(B),
        .Bin(Bin),
        .diff(diff),
        .Bout(Bout)
    );
    
    initial begin
        $dumpfile("fullSubtractor_tb.vcd");
        $dumpvars(0, fullSubtractor_tb);
        
        // Test all input combinations
        A = 0; B = 0; Bin = 0; #20;
        A = 0; B = 0; Bin = 1; #20;
        A = 0; B = 1; Bin = 0; #20;
        A = 0; B = 1; Bin = 1; #20;
        A = 1; B = 0; Bin = 0; #20;
        A = 1; B = 0; Bin = 1; #20;
        A = 1; B = 1; Bin = 0; #20;
        A = 1; B = 1; Bin = 1; #20;
        
        $display("Test Complete!");
        $finish;
    end
    
    initial begin
        $monitor("Time=%0d A=%b B=%b Bin=%b diff=%b Bout=%b",
                 $time, A, B, Bin, diff, Bout);
    end
endmodule
\end{lstlisting}

\section*{Truth Table and Results}
\begin{center}
\begin{tabular}{|c|c|c||c|c|}
\hline
A & B & Bin & diff & Bout \\
\hline
0 & 0 & 0 & 0 & 0 \\
0 & 0 & 1 & 1 & 1 \\
0 & 1 & 0 & 1 & 1 \\
0 & 1 & 1 & 0 & 1 \\
1 & 0 & 0 & 1 & 0 \\
1 & 0 & 1 & 0 & 0 \\
1 & 1 & 0 & 0 & 0 \\
1 & 1 & 1 & 1 & 1 \\
\hline
\end{tabular}
\end{center}

[Simulation results and waveform analysis will be added after running the testbench]

\end{document} 